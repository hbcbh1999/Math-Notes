\documentclass[a4paper, 11pt]{article}   	
\usepackage{geometry}       
\geometry{a4paper}
\geometry{margin=1in}	
\usepackage{paralist}
  \let\itemize\compactitem
  \let\enditemize\endcompactitem
  \let\enumerate\compactenum
  \let\endenumerate\endcompactenum
  \let\description\compactdesc
  \let\enddescription\endcompactdesc
  \pltopsep=\medskipamount
  \plitemsep=1pt
  \plparsep=5pt
\usepackage[english]{babel}
\usepackage[utf8]{inputenc}

\usepackage{bbm}
\usepackage{bm}
\usepackage{amsmath}
\usepackage{amssymb}
\usepackage{amsthm}
\usepackage{mathrsfs}
\usepackage{booktabs}
\usepackage{empheq}
\pagestyle{headings}
\newcommand{\boxwidth}{430pt}

\usepackage{fancyhdr}
\pagestyle{fancy}
\lhead{Numerical Methods for Differential Equations, 2017 Spring.}
\rhead{}

\title{\textbf{Lecture 5}}
\author{Zed}

\begin{document}
\maketitle
\section{Newton's Method for Solving Nonlinear Equations}
We want to find roots for nonlinear equation $g(x)=0$. 
\begin{itemize}
	\item[\textit{Algo.}] \textbf{Newton's Method}
	\begin{itemize}
		\item[$\cdot$] Initialize a starting point $x_0$. 
		\item[$\cdot$] Update $x$ by $x_{n+1} \leftarrow x_n - \frac{g(x_n)}{g'(x_n)}$, suppose $g'(x_n)\ne 0$.
		\item[$\cdot$] Stop when $\left\|g(x_n)\right\| \leq thres$ (small), $\left\|x_{n+1} - x_n\right\| \leq thres$ (small) or $n\geq K$ (fail to converge). 
	\end{itemize}
	The Newton's method only converges in a local sense. Suppose the real zero is $\alpha$, s.t. $g(\alpha)=0$, then using taylor expansion at $x_n$:
	\begin{equation}
		 \begin{split}
		 	0 = g(\alpha) &= g(x_n) + g'(x_n)(\alpha - x_n) + \frac{1}{2}g''(\xi_n)(\alpha - x_n)^2 \\
		 	\Rightarrow 0 &= \left(\frac{g(x_n)}{g'(x_n)} - x_n\right) + \alpha + \frac{g''(\xi_n)}{2g'(x_n)}(\alpha - x_n)^2 \\
		 \end{split}
	\end{equation}
	Hence exist bound $M$ such that 
	$$
	|\alpha - x_{n+1}| = \frac{g''(\xi_n)}{2g'(x_n)}|\alpha - x_n|^2 \leq M|\alpha - x_n|^2
	$$
	That is, if the error at $n$-th iteration $|\alpha - x_n|$ is already small, the next error at ($n+1$)-th iteration will be the square of it. Which implies a (locally) quadratic convergence rate.
\end{itemize}

\section{Implicit Methods}
\subsection{Implicit Runge-Kutta Method}
We consider the Runge-Kutta method with table:
\begin{center}
	\begin{tabular}{c|ccccc}
	$c_1$ & $a_{11}$ & $a_{12}$ & $\hdots$ & $a_{1,s-1}$ & $a_{1s}$\\
	$c_2$ & $a_{21}$ & $a_{22}$ & $\hdots$ & $a_{2,s-1}$ & $a_{2s}$\\
	$c_3$ & $a_{31}$ & $a_{32}$ & $\hdots$ & $a_{3,s-1}$ & $a_{3s}$\\
	$\vdots$ & $\vdots$ & $\vdots$ & $\ddots$ & $\vdots$ & $\vdots$ \\
	$c_s$ & $a_{s1}$ & $a_{s2} $ & $\hdots$ & $a_{s,s-1}$ & $a_{ss}$\\
	\hline
	 & $b_1$ & $b_2$ & $\hdots$ & $b_{s-1}$ & $b_s$
	\end{tabular} = 
	\begin{tabular}{c|ccc}
	1/2 & 1/2\\
	\hline
	 & 1 
	\end{tabular}
\end{center}
A table of this kind with a full matrix of entries $\bm{A}$ (instead of entries only in the lowertriangular area) suggests that this is an \emph{implicit} RK method. 

~\\
For the listed example, we have
\begin{equation}
	\begin{cases}
	y_{n+1} = y_n + hb_1 k_1 \\
	k_1 = f(t_n + c_1 h, y_n + ha_{11}k_1)
	\end{cases}
\end{equation}
And insert the values of coefficients:
$$
y_{n+1} = y_n + h\left(f(t_n+\tfrac{1}{2}h, \tfrac{y_{n+1}}{2}+\tfrac{y_n}{2})\right)
$$
Which is usually referred to as the \emph{Implicit Midpoint Method}. This method is the simplest implicit RK method, the simplest \emph{Gauss-Legendre} method. And a \emph{Symplectic method}, which is energy-preserving.

~\\
The Midpoint method is of order 2 (see the first question in HW1).
\begin{itemize}
	\item[\textit{Ex.}] (\emph{Energy Preserving}) Consider the Hamilton system: $p'=-q, q'=p$; with $p(0)=q(0)=1$. We can find a Hamilton function $H$ such that $p'=-\frac{\partial H}{\partial q}$, $q'=\frac{\partial H}{\partial p}$. And hence $\frac{\partial H}{\partial t}=0$. However, when we use the explicit RK method to solve the system (for example, the \texttt{ode45} in \texttt{Matlab}), we will find that the method does not preserve energy, i.e. $\frac{\partial H}{\partial t}\ne 0$.
\end{itemize}

\end{document}
