\documentclass[a4paper, 11pt]{article}   	
\usepackage{geometry}       
\geometry{a4paper}
\geometry{margin=1in}	
\usepackage{paralist}
  \let\itemize\compactitem
  \let\enditemize\endcompactitem
  \let\enumerate\compactenum
  \let\endenumerate\endcompactenum
  \let\description\compactdesc
  \let\enddescription\endcompactdesc
  \pltopsep=\medskipamount
  \plitemsep=1pt
  \plparsep=5pt
\usepackage[english]{babel}
\usepackage[utf8]{inputenc}

\usepackage{bbm}
\usepackage{bm}
\usepackage{amsmath}
\usepackage{amssymb}
\usepackage{amsthm}
\usepackage{mathrsfs}
\usepackage{booktabs}
\usepackage{empheq}
\pagestyle{headings}
\newcommand{\boxwidth}{430pt}

\usepackage{fancyhdr}
\pagestyle{fancy}
\lhead{Numerical Methods for Differential Equations, 2017 Spring.}
\rhead{}

\title{\textbf{Lecture 5}}
\author{Zed}

\begin{document}
\maketitle
\section{Newton's Method for Solving Nonlinear Equations}
We want to find roots for nonlinear equation $g(x)=0$. 
\begin{itemize}
	\item[\textit{Algo.}] \textbf{Newton's Method}
	\begin{itemize}
		\item[$\cdot$] Initialize a starting point $x_0$. 
		\item[$\cdot$] Update $x$ by $x_{n+1} \leftarrow x_n - \frac{g(x_n)}{g'(x_n)}$, suppose $g'(x_n)\ne 0$.
		\item[$\cdot$] Stop when $\left\|g(x_n)\right\| \leq thres$ (small), $\left\|x_{n+1} - x_n\right\| \leq thres$ (small) or $n\geq K$ (fail to converge). 
	\end{itemize}
	The Newton's method only converges in a local sense. Suppose the real zero is $\alpha$, s.t. $g(\alpha)=0$, then using taylor expansion at $x_n$:
	\begin{equation}
		 \begin{split}
		 	0 = g(\alpha) &= g(x_n) + g'(x_n)(\alpha - x_n) + \frac{1}{2}g''(\xi_n)(\alpha - x_n)^2 \\
		 	\Rightarrow 0 &= \left(\frac{g(x_n)}{g'(x_n)} - x_n\right) + \alpha + \frac{g''(\xi_n)}{2g'(x_n)}(\alpha - x_n)^2 \\
		 \end{split}
	\end{equation}
	Hence exist bound $M$ such that 
	$$
	|\alpha - x_{n+1}| = \frac{g''(\xi_n)}{2g'(x_n)}|\alpha - x_n|^2 \leq M|\alpha - x_n|^2
	$$
	That is, if the error at $n$-th iteration $|\alpha - x_n|$ is already small, the next error at ($n+1$)-th iteration will be the square of it. Which implies a (locally) quadratic convergence rate.
\end{itemize}
\section{Implicit Methods}
\subsection{Implicit Euler}
\subsection{Implicit Runge-Kutta Method}

\end{document}
