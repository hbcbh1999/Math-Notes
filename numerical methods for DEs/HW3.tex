\documentclass[a4paper, 10pt]{article}    
\usepackage{geometry}       
\geometry{a4paper}
\geometry{margin=1in} 
\usepackage{paralist}
  \let\itemize\compactitem
  \let\enditemize\endcompactitem
  \let\enumerate\compactenum
  \let\endenumerate\endcompactenum
  \let\description\compactdesc
  \let\enddescription\endcompactdesc
  \pltopsep=\medskipamount
  \plitemsep=1pt
  \plparsep=1pt
\usepackage[english]{babel}
\usepackage[utf8]{inputenc}

\usepackage{bbm, bm}
\usepackage{amsmath, amssymb, amsthm, mathrsfs}
\usepackage{booktabs, tikz}

\pagestyle{headings}
\newcommand{\boxwidth}{430pt}

\theoremstyle{definition}
\newtheorem{problem}{Problem}

\newtheoremstyle{hSol}
  {1.0pt}% Space above
  {1.0pt}% Space below
  {}% bodyfont
  {}% indent
  {\bfseries}% thm head font
  {.}% punctuation after thm head
  { }% Space after thm head
  {}% thm head spec

\theoremstyle{hSol}
\newtheorem*{solution}{Solution}



\title{\textbf{Numerical Solutions for DEs HW3}}
\author{YANG, Ze (5131209043)}


\begin{document}
\maketitle


%------------------------------------------------------------------------
~\\
\fbox{
  \parbox{\boxwidth}{
  \textbf{A Note to TA}: 
  ~\\
  \emph{Hi, this is the senior student from Antai College who did not register for this course. I would like to do all the assignments for practice, but feel free to just skip my homework if you don't have time.
  Thank you again for allowing me to access the assignments and other class material! : )}
  ~\\
  \emph{- Ze}
  }
}

~\\
~\\
%------------------------------------------------------------------------

\begin{problem} (Iserles 4.4) Determine all values of $\theta$ such that the theta method (1.13) is absolutely stable.
\end{problem}
\begin{proof} 

\end{proof}
\noindent\rule{16cm}{0.4pt}
%///////////////////////////////////////////////////////////////////////

\begin{problem} (Iserles 4.9) The two-step method 
$$
\bm{y}_{n+2} - \bm{y}_n = 2h \bm{f}(t_{n+1}, \bm{y}_{n+1}),~~~n=0,1,... 
$$ 
is called the explicit midpoint rule.
\begin{itemize}
  \item[a.] Denote by $w_1(z)$ and $w_2(z)$ the zeroes of the underlying function $\eta(z, \cdot)$, show that $w_1(z)w_2(z)\equiv -1$ for all $z\in \mathbb{C}$.
  \item[b.] Show that $\mathcal{D}=\emptyset$.
  \item[c.] We say that $\tilde{\mathcal{D}}$ is a \emph{weak linear stability domain} of a numerical method if, when applied to the scalar linear equation, it produces a uniformly bounded solution sequence. (It is easy to see that $\tilde{\mathcal{D}} = \text{cl}\mathcal{D}$ for most methods of interest.) Determine explicitly $\tilde{\mathcal{D}}$ for the method.
\end{itemize}
\end{problem}
\begin{proof} 

\end{proof} 
\noindent\rule{16cm}{0.4pt}
%///////////////////////////////////////////////////////////////////////


\begin{problem} (Iserles 5.8) Show that the \emph{H\'enon-Heiles system} (using the definition from wikipedia)
\begin{equation}
  \begin{split}
    q_1' &= p_1 \\
    p_1' &= -q_1 - 2 \lambda q_1 q_2\\
    q_2' &= p_2 \\
    p_2' &= -q_2 - \lambda (q_1^2 - q_2^2)
  \end{split}
\end{equation}
is Hamiltonian and identify explicitly the hamiltonian energy.
\end{problem}
\begin{proof} 

\end{proof} 
\noindent\rule{16cm}{0.4pt}
%///////////////////////////////////////////////////////////////////////

\begin{problem} The symplectic Euler method for the Hamiltonian system reads
\begin{equation}
\begin{cases}
  \bm{p}_{n+1} &= \bm{p}_n - h \frac{\partial H(\bm{p}_{n+1}, \bm{q}_n)}{\partial \bm{q}} \\
  \bm{q}_{n+1} &= \bm{q}_n + h \frac{\partial H(\bm{p}_{n+1}, \bm{q}_n)}{\partial \bm{p}}
\end{cases}
\end{equation}
\begin{itemize}
  \item[a.] Show that this is a first order method.
  \item[b.] Prove from basic principles that, as implied by its name, the method is indeed symplectic. (Hint: let $G=\nabla^2 H$, and write $G=[G_{11}, G_{12}; G_{21}, G_{22}]$).
  \item[c.] Assuming that the Hamiltonian is separable, $H(\bm{p}, \bm{q})=T(\bm{p})+V(\bm{q})$, where $T$ and $V$ correspond to kinetic and potential energy respectively. Show that the method can be implemented explicitly.
  \item[d.] Use symplectic Euler and explicit Euler to solve the problem of non-linear pendulum. The Hamiltonian is $H(p,q)=\frac{1}{2}p^2 - \cos({q})$, and the Hamiltonian equations are
  $$
  \dot{p} = -\sin q,~~~~\dot{q} = p
  $$
  with initial condition $p(0)=0$, $q(0)=1$. Plot the error of the numerical methods in the Hamiltonian $H$.
\end{itemize}
\end{problem}
\begin{proof} 

\end{proof} 
\noindent\rule{16cm}{0.4pt}
%///////////////////////////////////////////////////////////////////////

\begin{problem} Implement implicit Euler, implicit midpoint method and trapezoid method, compare their rates of convergence and execution time. 

\end{problem}
\begin{solution} 

\end{solution} 
\noindent\rule{16cm}{0.4pt}
%///////////////////////////////////////////////////////////////////////






\end{document}