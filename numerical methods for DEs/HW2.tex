\documentclass[a4paper, 10pt]{article}    
\usepackage{geometry}       
\geometry{a4paper}
\geometry{margin=1in} 
\usepackage{paralist}
  \let\itemize\compactitem
  \let\enditemize\endcompactitem
  \let\enumerate\compactenum
  \let\endenumerate\endcompactenum
  \let\description\compactdesc
  \let\enddescription\endcompactdesc
  \pltopsep=\medskipamount
  \plitemsep=1pt
  \plparsep=1pt
\usepackage[english]{babel}
\usepackage[utf8]{inputenc}

\usepackage{bbm, bm}
\usepackage{amsmath, amssymb, amsthm, mathrsfs}
\usepackage{booktabs, tikz}

\pagestyle{headings}
\newcommand{\boxwidth}{430pt}

\theoremstyle{definition}
\newtheorem{problem}{Problem}

\newtheoremstyle{hSol}
  {1.0pt}% Space above
  {1.0pt}% Space below
  {}% bodyfont
  {}% indent
  {\bfseries}% thm head font
  {.}% punctuation after thm head
  { }% Space after thm head
  {}% thm head spec

\theoremstyle{hSol}
\newtheorem*{solution}{Solution}



\title{\textbf{Numerical Solutions for DEs HW2}}
\author{YANG, Ze (5131209043)}


\begin{document}
\maketitle


%------------------------------------------------------------------------
~\\
\fbox{
  \parbox{\boxwidth}{
  \textbf{A Note to TA}: 
  ~\\
  \emph{Hi, this is the senior student from Antai College who did not register for this course. I would like to do all the assignments for practice, but feel free to just skip my homework if you don't have time.
  Thank you again for allowing me to access the assignments and other class material! : )}
  ~\\
  \emph{- Ze}
  }
}

~\\
~\\
%------------------------------------------------------------------------

\begin{problem}
\begin{itemize}
  \item[a.] Implement \emph{RK2}, \emph{Heun's Method} and the classical \emph{RK4}, and justify the rate of convergence numerically.
  \item[b.] Choose appropriate Runge-Kutta method to initialize \emph{Adams-Bashforth} method of order 3,
  $$
  y_{n+3} = y_{n+2} + h [\tfrac{23}{12}f(t_{n+2}, y_{n+2}) - \tfrac{4}{3} f(t_{n+1}, y_{n+1}) + \tfrac{5}{12}f(t_n, y_n)]
  $$
  Justify the rate of convergence numerically.
\end{itemize}
\end{problem}
\begin{proof} 
\end{proof} 
\noindent\rule{16cm}{0.4pt}
%///////////////////////////////////////////////////////////////////////

\begin{problem} Consider the RK method
$$
\frac{U_{n+1}-U_n}{\Delta t} = c_1k_1 + c_2 k_2
$$
where $k_1= f(t_n, U_n), k_2 = f(t_n+b_{21}\Delta t, U_n + b_{21}\Delta t k_1$, and $c_1, c_2, b_{21} \in \mathbb{R}$. 
\begin{itemize}
  \item[a.] Show that there is a choice of these parameters such that the truncation error of the method
  $$
  \frac{T_n}{\Delta t} = \frac{u(t_{n+1})-u(t_n)}{\Delta t} = c_1f(t_n, u(t_n)) + c_2 f(t_n+b_{21}\Delta t, u(t_n) + b_{21}\Delta t f(t_n, u(t_n)))
  $$
  is of order 2 as $\Delta t \to 0$.
  \item[b.] Suppose that a second order method of the above form is applied to the IVP $u'=-\lambda u$, $u(0)=1$, where $\lambda$ is real positive number. Show that the sequence $\{U_n\}_{n\geq 0}$ is bounded $\iff$ $\Delta t \leq 2/\lambda$ (\emph{interval of absolute stability}). Show further that, for such $\lambda$: $|u(t_n) - U_n|\leq \frac{1}{6}\lambda^3 \Delta t^2 t_n$, $n\geq 0$.
\end{itemize}
\end{problem}
\begin{proof} 
\end{proof} 
\noindent\rule{16cm}{0.4pt}
%///////////////////////////////////////////////////////////////////////

\begin{problem} Find the general solution $\{y_n\}$ for the homogeneous recurrence reltation
$$
a_0 y_n + a_1 y_{n+1} + ... + a_s y_{n+s} = 0
$$
\end{problem}
\begin{proof} 
\end{proof} 
\noindent\rule{16cm}{0.4pt}
%///////////////////////////////////////////////////////////////////////

\begin{problem} Find the region of A-stablity for the multistep method:
$$
U_{n+2} - U_n = \frac{1}{3}h [f(t_{n+2}, U_{n+2}) + 4f(t_{n+1}, U_{n+1}) + f(t_n, U_n) ]
$$
\end{problem}
\begin{proof} 
\end{proof} 
\noindent\rule{16cm}{0.4pt}
%///////////////////////////////////////////////////////////////////////

\begin{problem} (Iserles 3.4) Apply the classical RK4 method to scalar autonomous ode $y'=f(y)$, check that this method is indeed of order 4.
\end{problem}
\begin{proof} 
\end{proof} 
\noindent\rule{16cm}{0.4pt}
%///////////////////////////////////////////////////////////////////////

\begin{problem} (Iserles 3.5) Suppose that a $\nu$-stage RK method of order $\nu$ is applied to linear scalar ode $y' = \lambda y$, show that
$$
y_n = \left(\sum_{k=0}^{\nu} \frac{1}{k!} (h \lambda)^k\right)^n y_0,~~~~n=0,1,...
$$
\end{problem}
\begin{proof} 
\end{proof} 
\noindent\rule{16cm}{0.4pt}
%///////////////////////////////////////////////////////////////////////

\begin{problem} (Iserles 4.1) Let $\bm{y}'= \bm{\Lambda} \bm{y}$, $\bm{y}(t_0)=\bm{y}_0$ be solved (with a constant step size $h>0$) by a one-step method with a function $r$ that obeys the relation (4.12). Suppose that a nonsingular matrix $\bm{V}$ and a diagonal matrix $\bm{D}$ exists such that $\bm{\Lambda} = \bm{V} \bm{D} \bm{V}^{-1}$, show that there exists vectors $\bm{x}_1, \bm{x}_2, ..., \bm{x}_d \in \mathbb{R}^d$ such that
$$
\bm{y}(t_n) = \sum_{j=1}^d e^{t_n \lambda_j} \bm{x}_j,~~~~n=0,1,...
$$
And 
$$
\bm{y}_n = \sum_{j=1}^d [r(h \lambda)]^n \bm{x}_j,~~~~n=0,1,...
$$
where $\lambda_1, ..., \lambda_d$ are eigenvalues of $\bm{\Lambda}$. Deduce that the values of $\bm{x}_1$ and $\bm{x}_2$, given in (4.3) and (4.4) are identical.
\end{problem}
\begin{proof} 
\end{proof} 
\noindent\rule{16cm}{0.4pt}
%///////////////////////////////////////////////////////////////////////

\begin{problem} (Iserles 4.2) Consider the solution of $\bm{y}'=\bm{\Lambda} \bm{y}$ where 
$$
\bm{\Lambda} = \begin{pmatrix}
  \lambda & 1\\
  0 & \lambda
\end{pmatrix}, ~~~~\lambda \in \mathbb{C}^-
$$
\begin{itemize}
  \item[a.] Show that 
  $$
  \bm{\Lambda}^n = \begin{pmatrix}
    \lambda^n & n \lambda^{n-1} \\
    0 & \lambda^n
  \end{pmatrix},~~~~ n = 0,1,...
  $$
  \item[b.] Let $g$ be an arbitrary function that is analytic about the origin. The $2\times2$ matrix $g(\bm{\Lambda})$ can be defined by substituting powers of $\bm{\Lambda}$ into the Taylor expansion of $g$, show that
  $$
  g(t \bm{\Lambda}) = \begin{pmatrix}
    g(t \lambda) & tg'(t \lambda) \\
    0 & g(t \lambda)
  \end{pmatrix}
  $$
  \item[c.] By letting $g(z) = e^z$, show that $\lim\limits_{t\rightarrow\infty}\bm{y}(t)=\bm{0}$.
  \item[d.] Suppose that $\bm{y}'= \bm{\Lambda} \bm{y}$ is solved with RK method, using a constant step size $h>0$. Let $r$ be the function from lemma 4.1. Letting $g=r$, obtain the explicit form of $[r(h \bm{\Lambda})]^n$, $n=0,1,...$
  \item[e.] Show that if $h \lambda \in \mathcal{D}$, where $\mathcal{D}$ is the linaer stability domain of the RK method, then $\lim\limits_{n\rightarrow\infty} \bm{y}_n = \bm{0}$.
\end{itemize}
\end{problem}
\begin{proof} 
\end{proof} 
\noindent\rule{16cm}{0.4pt}
%///////////////////////////////////////////////////////////////////////










\end{document}