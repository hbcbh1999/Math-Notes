\documentclass[a4paper, 11pt]{article}   	
\usepackage{geometry}       
\geometry{a4paper}
\geometry{margin=1in}	
\usepackage{paralist}
  \let\itemize\compactitem
  \let\enditemize\endcompactitem
  \let\enumerate\compactenum
  \let\endenumerate\endcompactenum
  \let\description\compactdesc
  \let\enddescription\endcompactdesc
  \pltopsep=\medskipamount
  \plitemsep=1pt
  \plparsep=5pt
\usepackage[english]{babel}
\usepackage[utf8]{inputenc}

\usepackage{bbm}
\usepackage{bm}
\usepackage{amsmath}
\DeclareMathOperator*{\argmin}{argmin}
\DeclareMathOperator*{\argmax}{argmax}
\usepackage{amssymb}
\usepackage{amsthm}
\usepackage{mathrsfs}
\usepackage{booktabs}
\usepackage{empheq}
\pagestyle{headings}
\newcommand{\boxwidth}{430pt}

\usepackage{fancyhdr}
\pagestyle{fancy}
\lhead{Statistical Learning Theroy, 2017 Spring.}
\rhead{}

\title{\textbf{Linear Methods for Regression}}
\author{Zed}{}

\begin{document}
\maketitle

\section{Ordinary Least Squares}
We write the linear regression model
$$
f(X) = \beta_0 + \sum_{j=1}^p X_j \beta_j = X^{\top} \beta
$$
where $\beta=(\beta_0, \beta_1..., \beta_p)^{\top}$. $X=(1, X_1, ..., X_p)^{\top}$ is a $p+1$ column vector, with the inputs $X_j$ being quantitative, factor variables ($X_j = \mathbbm{1}_{\{G=\mathcal{G}_j\}}$), transformation of quantitative (say $\sin X_j$, $\log X_j$), basis expansions ($X_2 = X_1^2, X_3 = X_1^3$, ...) or cross terms ($X_3 = X_2 X_1$). We have a quick review of the familiar OLS estimator before proceeding to new concepts and models. 
\subsection{Algebraic Properties}
\begin{itemize}
	\item[\textit{Def.}] \textbf{Least Squares Estimator}: We choose sqaured error as loss function, and solve
	$$
	\hat{\beta} = \argmin\limits_{\beta} \sum_{i=1}^N (y_i - \bm{x}_i^{\top}\beta)^2 = \argmin\limits_{\beta} (\bm{y}-\bm{X}\beta)^{\top}(\bm{y}-\bm{X}\beta)
	$$
	by the familiar method of moments, and get $\hat{\beta} = (\bm{X}^{\top} \bm{X})^{-1}\bm{X}^{\top} \bm{y}$;
	~\\
\end{itemize}
the prediction for \emph{training set} is $\hat{\bm{y}}=\bm{X}(\bm{X}^{\top} \bm{X})^{-1}\bm{X}^{\top} \bm{y}$, which is, geometrically, an orthogonal projection of $\bm{y}$ onto the column space of $\bm{X}$, i.e. $\mathcal{C}(\bm{X})=\text{span}\{\text{Cols}(\bm{X})\}$. A few recap and highlights: 
\begin{itemize}
	\item[$\cdot$] (\emph{Orthogonal Projection}) $\hat{\bm{y}}$ is within $\mathcal{C}(\bm{X})$, since $\hat{y}=\bm{X}\hat{\beta}$, a linear combination of the columns of $\bm{X}$. The residual $\bm{y}-\hat{\bm{y}}$ is orthogonal to the subspace $\mathcal{C}(\bm{X})$, since $\bm{X}^{\top}(\bm{y}-\hat{\bm{y}}) = \bm{X}^{\top}(\bm{y}-\bm{X}(\bm{X}^{\top} \bm{X})^{-1}\bm{X}^{\top} \bm{y}) = 0$.

	\item[$\cdot$] (\emph{Orthogonal Complement}) Our sample $\bm{y} \in \mathbb{R}^N$, which can always be decomposed as $\mathbb{R}^N = V \oplus V^{\perp}$, where $V$ is a subspace, $V^{\perp}$ is the orthogonal complement of $V$. We already have the column space $\mathcal{C}(\bm{X})$, and we can show that $\mathcal{C}(\bm{X})^{\perp} = \mathcal{N}(\bm{X}^{\top})$, the null space of $\bm{X}^{\top}$, which has dimension $N-p-1$. \\
	\textit{Proof.~~} Suppose $\bm{z} \in \mathcal{C}(\bm{X})^{\perp}$, then $\bm{z}^{\top} \bm{X}\beta =0$ for all linear combination parameter $\beta \ne 0$. Hence the only way is $\bm{z}^{\top} \bm{X} = \bm{0}$, i.e. $\bm{X}^{\top} \bm{z} = \bm{0}$. $\square$ \\

	\item[$\cdot$] (\emph{Hat Matrix}) The matrix $\bm{H}_{\bm{X}} := \bm{X}(\bm{X}^{\top} \bm{X})^{-1}\bm{X}^{\top}$ is called the ``hat'' matrix, which maps a vector to its orthogonal projection on $\mathcal{C}(\bm{X})$. (symmetric, idempotent, and maps columns of $\bm{X}$ to itself.) A curious object is the trace of this matrix:
	$$
	\text{tr}(\bm{H}_{\bm{X}}) = \text{tr}(\bm{X}(\bm{X}^{\top} \bm{X})^{-1}\bm{X}^{\top}) = \text{tr}(\bm{I}_{p+1}) = p+1
	$$

	\item[$\cdot$] (\emph{Residual}) We are also interested in the error of the estimator \emph{within the training set}, i.e. define $\hat{\bm{u}}=\bm{y}-\hat{\bm{y}}$ as the residual term. It follows immediately that the residual sum of square $RSS= \hat{\bm{u}}^{\top} \hat{\bm{u}}$. And apply the hat matrix we see $\hat{\bm{u}}=(\bm{I}_N-\bm{H}_{\bm{X}})\bm{y}$. The object in between is also symmetric, idempotent, due to these property of $\bm{H}_{\bm{X}}$; consider
	$$
	(\bm{I}-\bm{H}_{\bm{X}})(\bm{I}-\bm{H}_{\bm{X}}) = \bm{I}-2 \bm{H}_{\bm{X}} + \bm{H}_{\bm{X}}
	$$

	\item[$\cdot$] (\emph{When $\bm{X}^{\top} \bm{X}$ is Singular}) When columns of $\bm{X}$ are linearly dependent, $\bm{X}^{\top} \bm{X}$ becomes singular, and $\hat{\beta}$ is not uniquely defined. But $\hat{\bm{y}}$ is still the orthogonal projection onto $\mathcal{C}(\bm{X})$, just with more than one way to do the projection.
\end{itemize}

\subsection{Statistical Properties}

(\textbf{Linear Assumptions}) To discuss statistical properties of $\hat{\beta}$, we assume that the linear model is the true model for the mean, i.e. the conditional expectation of $Y$ is $X\beta$, and that the devation of $Y$ from the mean is additive, distributed as $\epsilon \sim \mathcal{N}(0, \sigma^2)$. That is 
$$
Y=\mathbb{E}\left[Y\middle|X\right] + \epsilon = X\beta + \epsilon
$$
We further assume that the inputs $\bm{X}$ in the training set are fixed (non-random). 

Under these assumptions, a few other highlights on statistical properties of OLS estimator:
\begin{itemize}
	\item[$\cdot$] (\emph{Expectation of $\hat{\beta}$}) $\mathbb{E}(\hat{\beta}) = \mathbb{E}\left[(\bm{X}^{\top}\bm{X})^{-1} \bm{X}^{\top}(\bm{X}\beta + \epsilon)\right] = \beta$, i.e. it is an unbiased estimator.
	\item[$\cdot$] (\emph{Variance of $\hat{\beta}$}) $\mathrm{\mathbb{V}ar}(\hat{\beta}) = \mathbb{E}\left[(\bm{X}^{\top}\bm{X})^{-1} \bm{X}^{\top} \bm{\epsilon} \bm{\epsilon}^{\top}(\bm{X}^{\top} \bm{X})^{-1} \bm{X}\right] = \sigma^2 (\bm{X}^{\top} \bm{X})^{-1}$. That is, the estimator $\hat{\beta}\sim \mathcal{N}(\beta, \sigma^2(\bm{X}^{\top} \bm{X})^{-1})$
	\item[$\cdot$] (\emph{Residual Revisited}) With the assumption of the real model of $\bm{y}$, we can further write $\hat{\bm{u}}=(\bm{I}-\bm{H}_{\bm{X}})\bm{y} = (\bm{I}-\bm{H}_{\bm{X}})(\bm{X}\beta+\bm{\epsilon}) =(\bm{I}-\bm{H}_{\bm{X}}) \bm{\epsilon} $.
	It is easy to see that $\mathbb{E}\left[\hat{\bm{u}}\right] = \mathbb{E}\left[\bm{X}(\beta-\hat{\beta})+\bm{\epsilon}\right] = 0$. And therefore
	$$
	\mathrm{\mathbb{V}ar}\left[\hat{\bm{u}}\right] = \mathbb{E}[\hat{\bm{u}}\hat{\bm{u}}^{\top}] = \mathbb{E}\left[(\bm{I}-\bm{H}_{\bm{X}}) \bm{\epsilon} \bm{\epsilon}^{\top}(\bm{I}-\bm{H}_{\bm{X}})\right] = \sigma^2 (\bm{I}-\bm{H}_{\bm{X}})
	$$
	So, although the errors $\bm{\epsilon}$ are i.i.d., residuals $\hat{\bm{u}}$ are correlated. 
	\item[$\cdot$] (\emph{Individual Residual Term}) Pick any individual residual $\hat{u}_i$, $\mathrm{\mathbb{V}ar}\left[\hat{u}_i\right] = \sigma^2(1-h_i)$, where $h_i$ is the i-th diagonal entry of $\bm{H}_{\bm{X}}$. Furthermore $\mathrm{\mathbb{C}ov}\left[\hat{u}_i, \hat{u}_j\right] = \sigma^2 h_{ij}$, $i\ne j$, $h_{ij}$ is the row $i$, column $j$ entry in $\bm{H}_{\bm{X}}$.
\end{itemize}

~\\
An unbiased estimator of residual variance (square of residual standard error: $RSE^2$) is
$$
\hat{\sigma}^2 = \frac{RSS}{N-p-1} = \frac{\hat{\bm{u}}^{\top} \hat{\bm{u}}}{N-p-1}
$$
\begin{itemize}
	\item[\textit{Prop.}] $\mathbb{E}(\hat{\sigma}^2) = \sigma^2$. We present two proofs. \\
	\textit{Proof (1).~~} 
	\begin{equation}
		\begin{split}
			\mathbb{E}\left[\hat{\bm{u}}^{\top} \hat{\bm{u}}\right] = \mathbb{E}\left[\sum_{i=1}^N \hat{u}_i^2\right] = \sum_{i=1}^N \mathrm{\mathbb{V}ar}\left[\hat{u}_i\right] = \sum_{i=1}^N \sigma^2(1-h_i)
		\end{split}
	\end{equation}
	By the trace formula we have discussed, $\sum{h_i} = \text{tr}(\bm{H}_{\bm{X}}) = p+1$. Hence $(2)=\sigma^2(N-p-1)$. We conclude that 
	$$
	(N-p-1)\mathbb{E}(\hat{\sigma}^2) = \mathbb{E}\left[\bm{\epsilon}^{\top}(\bm{I}-\bm{H}_{\bm{X}})\bm{\epsilon}\right] = (N-p-1)\sigma^2~~~~~\square.
	$$
	Before the second proof, we present a lemma.
	\item[\textit{Lemma.}] (\textbf{Distribution of Quadratic Form})
	\begin{itemize}
		\item[$\cdot$] If an $n$-vector $\bm{x}$ is distributed as $\mathcal{N}(\bm{0}, \bm{\Sigma})$, then the quadratic form ${\bm{x}}^{\top} \bm{\Sigma}^{-1} {\bm{x}} \sim \chi^2(n)$.
		\item[$\cdot$] If an $n$-vector $\bm{x}$ is standard multivariate normal: $\mathcal{N}(\bm{0}, \bm{I})$, and $\bm{H}_{\bm{Z}}$ is a projection matrix onto the column space of $\bm{Z}$, which has dimension $r$ (i.e. consider $\bm{Z}$ is a $n\times r$ matrix, and $\bm{Z}$ and $\bm{H}_{\bm{Z}}$ both have rank $r$); then the quadratic form $\bm{x}^{\top} \bm{H}_{\bm{Z}} \bm{x}\sim \chi^2(r)$.
	\end{itemize}

	\textit{Proof of lemma.~~} (\emph{First Part}) Since $\bm{\Sigma}$ is symmetric positive definite, we have \emph{Cholesky decomposition} $\bm{\Sigma} = \bm{Q}\bm{Q}^{\top}$, where $\bm{Q}$ is $n\times n$ lower triangular.
	$$
	\bm{x}^{\top} \bm{\Sigma}^{-1}\bm{x} = \bm{x}^{\top} \bm{Q}^{-\top}\bm{Q}^{-1}\bm{x} = (\bm{Q}^{-1}\bm{x})^{\top}(\bm{Q}^{-1}\bm{x}) = \bm{z}^{\top} \bm{z}
	$$
	in which we let $\bm{z}:=\bm{Q}^{-1}\bm{x}$. It is clear that $\mathbb{E}\left[\bm{z}\right] = \bm{Q}^{-1} \mathbb{E}\left[\bm{x}\right] = 0$. And 
	$$
	\mathrm{\mathbb{V}ar}\left[\bm{z}\right] = \mathbb{E}\left[\bm{\bm{Q}^{-1}\bm{x}}(\bm{Q}^{-1}\bm{x})^{\top}\right] = \bm{Q}^{-1} \mathbb{E}[\bm{x}\bm{x}^{\top}] \bm{Q}^{-\top} = \bm{Q}^{-1} \mathrm{\mathbb{V}ar}\left[\bm{x}\right] \bm{Q}^{-\top} = \bm{Q}^{-1} \bm{\Sigma} \bm{Q}^{-\top} = \bm{I}
	$$
	which indicates that $\bm{z}\sim \mathcal{N}(\bm{0}, \bm{I}_n)$ is an $n$-variate standard normal. It follows that $\bm{z}^{\top} \bm{z}\sim \chi^2(n)$. ~~~$\square$ \\
	(\emph{Second Part}) 
	$$
	\bm{x}^{\top} \bm{H}_{\bm{Z}}\bm{x} = \bm{x}^{\top} \bm{Z}(\bm{Z}^{\top} \bm{Z})^{-1}\bm{Z}^{\top}\bm{x} = \bm{y}^{\top} \bm{\Omega}^{-1}\bm{y}
	$$
	in which we let $\bm{y}:=\bm{Z}^{\top} \bm{x}$ (an $r\times 1$ vector), and $\bm{\Omega}:=\bm{Z}^{\top} \bm{Z}$ (an $r\times r$ matrix). This is exactly the form in part 1. And the linear transform of $n$-variate normal: $\bm{Z}^{\top} \bm{x}$ is distributed as $r$-variate normal $\mathcal{N}(\bm{0}, \bm{Z}^{\top} \bm{Z})$. By the result of part 1 $\Rightarrow \bm{x}^{\top} \bm{H}_{\bm{Z}}\bm{x} \sim \chi^2(r)$. $\square$


	\textit{Proof (2).~~}
	\begin{equation}
		\begin{split}
			(N-p-1)\hat{\sigma}^2 &= \hat{\bm{u}}^{\top} \hat{\bm{u}} = \bm{y}^{\top} (\bm{I}-\bm{H}_{\bm{X}})^{\top}(\bm{I}-\bm{H}_{\bm{X}})\bm{y} \\
			&= \bm{\epsilon}^{\top}(\bm{I}-\bm{H}_{\bm{X}})\bm{\epsilon} = \bm{\epsilon}^{\top} \bm{H}_{\bm{Z}} \bm{\epsilon}
		\end{split}
	\end{equation}
	in which we let $\bm{H}_{\bm{Z}}:= \bm{I}-\bm{H}_{\bm{X}}$. By previous result, this is also symmetric, idempotent, and projects any vector to the null space of $\bm{X}^{\top}$, the orthogonal complement of $\mathcal{C}(\bm{X})$. We can always compose a matrix $\bm{Z}$ whose columns are the general solutions of $\bm{X}^{\top} \bm{z} = 0$. Clearly it has $N-p-1$ columns, since the orthogonal complement has dimension $N-p-1$. Hence $\bm{H}_{\bm{Z}}$ has $(N-p-1)$ rank. Morever, $\bm{\epsilon}^{\top} \bm{H}_{\bm{Z}} \bm{\epsilon} = \bm{\epsilon}^{\top}\bm{Z}(\bm{Z}^{\top} \bm{Z})^{-1}\bm{Z}^{\top} \bm{\epsilon}$, and $\bm{Z}^{\top} \bm{Z}$ is of $(N-p-1)\times(N-p-1)$. By lemma, and multiply a normalization factor $\Rightarrow$ $\bm{Z}^{\top} \bm{\epsilon}/\sigma \sim \mathcal{N}(\bm{0}, (\bm{Z}^{\top} \bm{Z}))$, $\frac{1}{\sigma^2}\bm{\epsilon}^{\top} \bm{H}_{\bm{Z}} \bm{\epsilon} \sim \chi^2(N-p-1)$. So:
	$$
		\mathbb{E}\left[\bm{\epsilon}^{\top} \bm{H}_{\bm{Z}} \bm{\epsilon}\right] = \sigma^2(N-p-1)~~~~\square
	$$
	Proof (2) gives us a stronger result:
	\item[\textit{Prop.}] (\emph{Distribution of Sample Estimator of Variance}) The residual sum of square is Chi squared distributed with degree of freedom $(N-p-1)$.
	$$
	(N-p-1)\hat{\sigma}^2 = RSS \sim \sigma^2 \chi^2(N-p-1)
	$$
	In addition, $\hat{\beta}$ and $\hat{\sigma}$ are independent.
\end{itemize}

\subsection{Hypothesis Tests}
(\textbf{t Statistic}) The $t(n)$ distribution is defined as $t(n)\sim \frac{\mathcal{N}(0,1)}{\sqrt{\chi^2(n)/n}}$. To test hypothesis that a particular coefficient $\beta_j=0$, we formulate the statistic
$$
t_j = \frac{\hat{\beta}_j/\mathrm{se}(\hat{\beta_j})}{\sqrt{(N-p-1)\hat{\sigma}^2/(N-p-1)\sigma^2}} = \frac{\hat{\beta}_j}{\hat{\sigma} \cdot \mathrm{se}(\hat{\beta}_j)/\sigma} = \frac{\hat{\beta}_j}{\hat{\sigma} \sqrt{v_j}}
$$
where $\hat{\sigma}=\sqrt{RSS/(N-p-1)}$, $\sqrt{v_j}$ is the $j$-th diagonal element of $(\bm{X}^{\top} \bm{X})^{-1}$. And we know that $\hat{\beta}_j/\mathrm{se}(\hat{\beta_j})\sim \mathcal{N}(\beta_j/\mathrm{se}(\hat{\beta_j}),1)$ and that $\sqrt{(N-p-1)\hat{\sigma}^2/(N-p-1)\sigma^2}\sim \sqrt{\chi^2_{N-p-1}/(N-p-1)}$. Under the null hypothesis $\beta_j=0$, $\hat{\beta}_j/\mathrm{se}(\hat{\beta_j})\sim \mathcal{N}(0,1)$. We have $t_j \sim t(N-p-1)$. 
~\\ 
If we know $\sigma$ before hand, we just use it instead of $\hat{\sigma}$. And $t_j$ reduces to $\hat{\beta}_j/\mathrm{se}(\hat{\beta_j})\sim \mathcal{N}(0,1)$. Where $\mathrm{se}(\hat{\beta_j}) = \sigma \sqrt{v_j}$.

~\\
(\textbf{F Statistic}) The $\mathcal{F}(n_1, n_2)$ distribution is defined as $\mathcal{F}(n_1, n_2)\sim \frac{\chi^2(n_1)/n_1}{\chi^2(n_2)/n_2}$. To test hypothesis that $k$ coefficients $\beta_{[1]}=...=\beta_{[k]}=0$ simultaneously, we formulate the statistic
$$
F = \frac{(RSS_0 - RSS_1)/p_1-p_0}{RSS_1/(N-p_1-1)}
$$
Where the bigger model 1 has $p_1+1$ parameters, the smaller model 0 (corresponds to null hypothesis $H_0$) has $p_0+1$ parameters, $p_1-p_0=k$. We have $F\sim \mathcal{F}(p_1-p_0, N-p_1-1)$ under the null hypothesis.

~\\
(\textbf{Confidence Interval}) We can isolate $\beta_j$ to form a $1-2\alpha$ confidence interval
$$
\beta_j \in (\hat{\beta}_j-z_{(1-\alpha)} \sqrt{v_j}\hat{\sigma}, \hat{\beta}_j+z_{(1-\alpha)} \sqrt{v_j}\hat{\sigma})
$$
\textit{Proof.~~} We know that $\hat{\beta}\sim \mathcal{N}(\beta, \sigma^2 (\bm{X}^{\top} \bm{X})^{-1})$, a multivariate normal. So isolating $\hat{\beta}_j$, we have $\hat{\beta}_j\sim \mathcal{N}(\beta_j, \sigma^2 v_j)$, where, as before, $v_j$ is the j-th diagonal element of the covariance matrix of $\hat{\beta}$. $\text{se}(\hat{\beta}_j)=\sigma \sqrt{v_j}$. And hence $\frac{\hat{\beta}_j-\beta_j}{\sigma \sqrt{v_j}}\sim \mathcal{N}(0,1)$.
$$
1- 2 \alpha = \mathbb{P}\left(\left|\frac{\hat{\beta}_j-\beta_j}{\sigma \sqrt{v_j}}\right|> z_{(1-\alpha)}\right) = \mathbb{P}\left(\hat{\beta}_j-z_{(1-\alpha)} \sqrt{v_j}\sigma<\beta_j<\hat{\beta}_j+z_{(1-\alpha)} \sqrt{v_j}\sigma\right)
$$
And substitute $\sigma$ with the estimate $\hat{\sigma}$, yields the result. $\square$

~\\
(\textbf{Confidence Region}) We also obtain a confidence set for the entire parameter vector $\beta$, 
$$
\beta \in C_{\beta} = \{(\hat{\beta}-\beta)^{\top}\bm{X}^{\top} \bm{X}(\hat{\beta}-\beta)\leq \hat{\sigma}^2 \chi^2_{p+1, (1-\alpha)}\}
$$ 
\textit{Proof.~~} We know $\hat{\beta}-\beta \sim \mathcal{N}(\bm{0}, \sigma^2 (\bm{X}^{\top} \bm{X})^{-1})$, by \emph{lemma} (Dist of quadratic form) part 1, 
$(\hat{\beta}-\beta)^{\top} \frac{1}{\sigma^2}(\bm{X}^{\top} \bm{X}) (\hat{\beta}-\beta) \sim \chi^2(p+1)$. Hence
$$
1- \alpha = \mathbb{P}\left((\hat{\beta}-\beta)^{\top} \frac{1}{\sigma^2}(\bm{X}^{\top} \bm{X}) (\hat{\beta}-\beta) \leq \chi^2_{p+1, (1-\alpha)} \right) = \mathbb{P}\left((\hat{\beta}-\beta)^{\top} (\bm{X}^{\top} \bm{X}) (\hat{\beta}-\beta) \leq \sigma^2\chi^2_{p+1, (1-\alpha)} \right)
$$
And substitute $\sigma$ with the estimate $\hat{\sigma}$, yields the result. $\square$

\subsection{Gauss Markov Theorem}
\begin{itemize}
	\item[\textit{Thm.}] (\textbf{Gauss-Markov}) the least squares estimator has smallest variance among all \emph{linear unbiased} estimates.

	\textit{Proof.~~} Let $\tilde{\beta}$ be an unbiased linear estimator other than $\hat{\beta}$, which is the ols estimator. By linearity: $\tilde{\beta} = \bm{A} \bm{y}$, where $\bm{A}$ is some (non-random) matrix. Hence we may decompose $\tilde{\beta} = ((\bm{X}^{\top} \bm{X})^{-1}\bm{X}^{\top} + \bm{C})\bm{y} = \hat{\beta} + \bm{Cy}$, where we let $\bm{C}:=\bm{A} - (\bm{X}^{\top} \bm{X})^{-1}\bm{X}^{\top}$. 

	~\\
	By unbiasedness: $\beta=\mathbb{E}[\tilde{\beta}] = \mathbb{E}\left[\bm{Ay}\right] = \mathbb{E}\left[\bm{A}(\bm{X}\beta + \bm{\epsilon})\right] = \bm{AX} \beta + \bm{A} \mathbb{E}\left[\bm{\epsilon}\right]$. Since the last term has mean $\bm{0}$, this requires $\bm{AX}=\bm{I}$ $\Rightarrow \bm{CX}=\bm{O}$. Hence $\bm{Cy} = \bm{C}(\bm{X}\beta + \bm{\epsilon}) = \bm{C \epsilon}$. Therefore
	\begin{equation}
		\begin{split}
			\mathrm{\mathbb{C}ov}[\hat{\beta}, \bm{Cy}] &= \mathrm{\mathbb{C}ov}[\hat{\beta}, \bm{C \epsilon}] = \mathbb{E}[(\hat{\beta} - \mathbb{E}\hat{\beta})(\bm{C \epsilon} - \bm{C}\mathbb{E}\bm{\epsilon})^{\top}] = \mathbb{E}[(\hat{\beta} - \beta)\bm{\epsilon}^{\top} \bm{C}^{\top}]  \\
			&= \mathbb{E}[(\bm{X}^{\top}\bm{X})^{-1} \bm{X}^{\top} \bm{\epsilon}\bm{\epsilon}^{\top} \bm{C}^{\top}] = \sigma^2 (\bm{X}^{\top} \bm{X})^{-1} (\bm{CX})^{\top} = \bm{O}
		\end{split}
	\end{equation}

	So: 
	$$
		\mathrm{\mathbb{V}ar}[\tilde{\beta}]  = \mathrm{\mathbb{V}ar}[\hat{\beta} + \bm{Cy}] = \mathrm{\mathbb{V}ar}[\hat{\beta} + \bm{C \bm{\epsilon}}] = \mathrm{\mathbb{V}ar}[\hat{\beta}] + \sigma^2 \bm{C}\bm{C}^{\top}~~~~~\square
	$$

	\subsection{Algorithm for Multiple Regression}
	For the univariate regression (with no intercept), we calculate ols estimator as:
	$$
	\hat{\beta}_1 = (\bm{x}^{\top}\bm{x})^{-1} \bm{x}^{\top} \bm{y} = \frac{\langle \bm{x}, \bm{x} \rangle}{\langle \bm{x}, \bm{y} \rangle}
	$$
	And the residual $\bm{r} = \bm{y} - \bm{x}\hat{\beta}$. Suppose $\langle \bm{x}_i, \bm{x}_j \rangle = 0$, i.e. $\bm{X}$ is an orthogonal matrix, then $\hat{\beta}_j = \langle \bm{x}_j, \bm{y} \rangle/ \langle \bm{x}_j, \bm{x}_j \rangle$, just write down $(\bm{X}^{\top}\bm{X})^{-1} \bm{X}^{\top} \bm{y}$ and use the fact that $\bm{X}$ is orthogonal we can easily get the result. This implies that when the inputs are orthogonal, they have no effect on each other's parameter estimates in the model.

	~\\ 
	For non-orthogonal $\bm{X}$, we perform the \emph{Gram-Schmidt} orthogonalization procedure:
	\begin{itemize}
		\item[\textit{Algo.}] (\emph{Gram-Schmidt}) Suppose $\bm{X}=(\bm{1}, \bm{x}_1, ..., \bm{x}_p)$.
		\begin{itemize}
			\item[1.] Let $\bm{z}_0 \leftarrow  \bm{x}_0 \leftarrow  \bm{1}$.
			\item[2.] \texttt{For j = 1:p}: Regress $\bm{x}_j$ on $\bm{z}_0, ..., \bm{z}_{j-1}$ respectively to produce coefficients $\hat{\gamma}_{ij} \leftarrow \langle \bm{z}_i, \bm{x}_j \rangle / \langle \bm{z}_i, \bm{z}_i \rangle$, $i=0,1,...,j-1$; $\hat{\gamma}_{jj} \leftarrow 1$.
			\item[3.] Calculate residual $\bm{z}_j \leftarrow \bm{x}_j - \sum_{i=0}^{j-1} \hat{\gamma}_{ij} \bm{z}_i$ 
			\item[4.] Regress $\bm{y}$ on the residual $\bm{z}_j$ to produce $\hat{\beta}_j \leftarrow \langle \bm{z}_j, \bm{y} \rangle / \langle \bm{z}_j, \bm{z}_j \rangle$
		\end{itemize}

		\item[\textit{Prop.}] $\bm{Z} = (\bm{z}_0, \bm{z}_1..., \bm{z}_p)$ is orthogonal. \\
		\textit{Proof.~~} We show by induction proof. Firstly, it is easy to see that 
		$$
		\langle \bm{z}_0, \bm{z}_1 \rangle = \langle \bm{z}_0, \bm{x}_1 - \frac{\langle \bm{z}_0, \bm{x}_1 \rangle}{\langle \bm{z}_0, \bm{z}_0 \rangle} \bm{z}_0 \rangle = \langle \bm{z}_0, \bm{x}_1 \rangle - \langle \bm{z}_0, \bm{x}_1 \rangle = 0
		$$
		We assume $\langle \bm{z}_0, \bm{z}_k \rangle = 0$ for all $1 < k\leq j < p$. Then for $k=j+1$: 
		$$
		\langle \bm{z}_0, \bm{z}_{j+1} \rangle = \langle \bm{z}_0, \bm{x}_{j+1}-\sum_{l=0}^j\frac{\langle \bm{z}_l, \bm{x}_{j+1} \rangle}{\langle \bm{z}_l, \bm{z}_l \rangle} \bm{z}_l\rangle = \langle \bm{z}_0, \bm{x}_{j+1} \rangle - \langle \bm{z}_0, \frac{\langle \bm{z}_0, \bm{x}_{j+1} \rangle}{\langle \bm{z}_0, \bm{z}_0\rangle} \bm{z}_0 \rangle = 0
		$$
		So we conclude that $\langle \bm{z}_0, \bm{z}_j \rangle = 0$ for $j=1,2,...,p$. Do the same induction for $\bm{z}_1$ as follows:
		\begin{itemize}
			\item[$\cdot$] Base case, using the fact (what we already known): $\langle \bm{z}_0, \bm{z}_1 \rangle = 0$
			$$
			\langle \bm{z}_1, \bm{z}_{2} \rangle = \langle \bm{z}_1, \bm{x}_{2}-\frac{\langle \bm{z}_0, \bm{x}_{2} \rangle}{\langle \bm{z}_0, \bm{z}_0\rangle} \bm{z}_0 - \frac{\langle \bm{z}_1, \bm{x}_{2} \rangle}{\langle \bm{z}_1, \bm{z}_1\rangle} \bm{z}_1\rangle = \langle \bm{z}_1, \bm{x}_2 \rangle - \langle \bm{z}_1, \bm{x}_2 \rangle = 0
			$$ 
			\item[$\cdot$] The induction, assume $\langle \bm{z}_1, \bm{z}_k \rangle = 0$ for all $2 < k\leq j < p$. Then for $k=j+1$: 
			$$
			\langle \bm{z}_1, \bm{z}_{j+1} \rangle = \langle \bm{z}_1, \bm{x}_{j+1}-\sum_{l=0}^j\frac{\langle \bm{z}_l, \bm{x}_{j+1} \rangle}{\langle \bm{z}_l, \bm{z}_l \rangle} \bm{z}_l\rangle = \langle \bm{z}_1, \bm{x}_{j+1} \rangle - \langle \bm{z}_1, \frac{\langle \bm{z}_1, \bm{x}_{j+1} \rangle}{\langle \bm{z}_1, \bm{z}_1\rangle} \bm{z}_1 \rangle = 0
			$$
		\end{itemize}
		So we conclude that $\langle \bm{z}_1, \bm{z}_j \rangle = 0$ for $j=2,...,p$. And the induction for $\bm{z}_i$, $i=2,3,...,p-1$ in the same fashion, we have $\bm{Z}$ is orthogonal. $\square$
	\end{itemize}
	Another observation is that $\bm{x}_j$ is a linear combination of $\bm{z}_k$, for $k\leq j$. Hence $\bm{Z}$ is a orthogonal basis for the column space of $\bm{X}$. Let $\bm{D}=\text{diag}(\left\|\bm{z}_j\right\|)$, then $\bm{Z}\bm{D}^{-1}$ gives the \emph{orthonormal basis} of column sapce of $\bm{X}$. We denote $\bm{Q}:=\bm{Z}\bm{D}^{-1}$, which is also an orthogonal matrix.

	~\\
	By writing the algo in a matrix form, we denote $\bm{\Gamma}=\{\hat{\gamma}_{ij}\}$, which is an upper triangular matrix with main diagonal entries being $1$s. And hence we have
	$$
	\bm{X} = \bm{Z\Gamma} = \bm{ZD}^{-1} \bm{D\Gamma} =: \bm{QR}
	$$
	And the ols estimator given by
	$$
	\hat{\beta} = (\bm{X}^{\top}\bm{X})^{-1} \bm{X}^{\top} \bm{y} = (\bm{R}^{\top}\bm{Q}^{\top}\bm{QR})^{-1} \bm{R}^{\top} \bm{Q}^{\top} \bm{y} = \bm{R}^{-1} \bm{R}^{-\top} \bm{R}^{\top} \bm{Q}^{\top} \bm{y} = \bm{R}^{-1} \bm{Q}^{\top} \bm{y}
	$$
	$$
	\hat{\bm{y}} = \bm{X}\hat{\beta} = \bm{QR}\bm{R}^{-1} \bm{Q}^{\top} \bm{y} = \bm{Q}\bm{Q}^{\top} \bm{y}
	$$
\end{itemize}


\section{Subset Selection}
\begin{itemize}
	\item[$\cdot$] (\emph{Best-Subset Selection}) Look at all possible models at every given number ($k$) of variables chosen. (computationally expensive, becomes infeasible for $p$ much larger than 30-40 or so)
	\item[$\cdot$] (\emph{Forward-Stepwise Selection}) Rather than search through all possible subsets, we want to seek a path through them. FSS proceeds by sequentially adds into the model the predictor that most improves the fit. This is charactered as a \emph{greedy algorithm}, which must produce a nested sequence of models, i.e. it may not find the best model, when, for example, the best subset of size 2 does not include that of size 1 (which may happen). However, it has lower variance compared with best-subset.
	\item[$\cdot$] (\emph{Backward-Stepwise Selection}) Starts with the full model, and sequentially deletes the predictors that has the least impact on the fit. Can only be used for $N>p$. 
	\item[$\cdot$] (\emph{Forward-Stagewise (FS) Selection}) Start as the forward-stepwise, with intercept $\bar{y}$, and centered predictors with coefficients initially set as $0$. Then at each step, choose the variable that are most \emph{correlated} with the current residual, then compute simple regression param $\gamma$ of residual on this varible, add this to the current $\beta_j$, i.e. $\beta_j \leftarrow \beta_j + \gamma$. Continues until none are correlated with the residual. The convergence of this algorithm can be slow, but it has good performance for problems with high dimensionality.
\end{itemize}
Subset selection is a \emph{discrete} process, we either include a variable or exclude it. As a result it often exhibits high variance. Shrinkage methods are more continuous, and do not suffer from high variability.

\section{Shrinkage Methods}
The motivation of various shrinkage methods is to overcome the \emph{combinatorial explosion} of the number of possible subsets (when $p$ large) by converting the discrete problem to a continuous one, which turn out to be simpler to solve.
\subsection{Ridge Regression}
The ridge regression shrinks the regression coefs by imposing a penalty on the magnitudes of these coefficients. Denote the ridge estimator $\hat{\beta}^{ridge}$, it minimizes a penalized sum of squares:

$$
\hat{\beta}^{ridge} = \argmin\limits_{\beta} \left\{\sum_{i=1}^N (y_i - \beta_0 - \sum_{j=1}^p x_{ij}\beta_j)^2 + \lambda \sum_{j=1}^p \beta_j^2\right\}~~~~(*)
$$
where $\lambda$ is a hyperparameter (complexity parameter) that controls the amount of shrink. The solution of $(*)$ are not equivalent under the scaling of inputs. Hence we usually standardize the input before solving $(*)$. In addition, we don't penalize the magnitude $\beta_0$

~\\
The standardization procedure is done as: calculate the centered inputs as $x_{ij}-\bar{x}_j$, (in the following text we assume $\{\bm{X}\}_{ij}$ is this, has $p$ columns without $\bm{1}$), and estimate $\hat{\beta}_0$ by $\sum_{1}^N y_i/N$.
~\\
\begin{itemize}
	\item[\textit{Def.}] \textbf{Ridge Regression Estimator}: We minimize loss function with penalization:
	$$
	\hat{\beta}^{ridge} = \argmin\limits_{\beta} \{(\bm{y} - \bm{X}\beta)^{\top}(\bm{y} - \bm{X}\beta) + \lambda \beta^{\top} \beta\}
	$$
	We have $\partial RSS(\lambda)/\partial \beta = 2\bm{X}^{\top}(\bm{y} - \bm{X}\beta) + 2 \lambda \beta = 0$ $\Rightarrow$ $\hat{\beta}^{ridge} = (\bm{X}^{\top}\bm{X}+\lambda \bm{I}_p)^{-1} \bm{X}^{\top} \bm{y}$
\end{itemize}
Note that $\hat{\beta}^{ridge}$ is still a linear function of $\bm{y}$. And $\bm{X}^{\top}\bm{X}+\lambda \bm{I}_p$ is nonsigular, even if $\bm{X}^{\top} \bm{X}$ is singular.



\end{document}
