\documentclass[a4paper, 11pt]{article}   	
\usepackage{geometry}       
\geometry{a4paper}
\geometry{margin=1in}	
\usepackage{paralist}
  \let\itemize\compactitem
  \let\enditemize\endcompactitem
  \let\enumerate\compactenum
  \let\endenumerate\endcompactenum
  \let\description\compactdesc
  \let\enddescription\endcompactdesc
  \pltopsep=\medskipamount
  \plitemsep=1pt
  \plparsep=5pt
\usepackage[english]{babel}
\usepackage[utf8]{inputenc}

\usepackage{bbm}
\usepackage{bm}
\usepackage{amsmath}
\usepackage{amssymb}
\usepackage{amsthm}
\usepackage{mathrsfs}
\usepackage{booktabs}

\pagestyle{headings}
\newcommand{\boxwidth}{430pt}

\title{\textbf{Eigenvalues and Eigenvectors}}
\author{Zed}

\begin{document}
\maketitle
\section{}
\subsection{Subspaces}
\begin{itemize}
  \item[$\cdot$] $\mathbb{R}$, $\mathbb{C}$ field of real and complex numbers.
  \item[$\cdot$] Colspace of $\bm{A}^{m\times n}$: 
  $$
  \mathcal{C}(\bm{A}):=\text{span}\left\{\text{Cols}(\bm{A})\right\}\subseteq \mathbb{R}^n
  $$
  And $\text{Dim}(\mathcal{C}(\bm{A}))=r$. $r$ is rank of $\bm{A}$.

  \item[$\cdot$] Rowspace of $\bm{A}^{m\times n}$: 
  $$
  \mathcal{R}(\bm{A}):=\text{span}\left\{\text{Rows}(\bm{A})\right\}=\mathcal{C}(\pmb{A}^{\top})\subseteq \mathbb{R}^m
  $$ 
  And $\text{Dim}(\mathcal{R}(\bm{A}))=r$.

  \item[$\cdot$] Nullspace of $\bm{A}^{m\times n}$: 
  $$
  \mathcal{N}(\bm{A}):=\{\bm{x}: \bm{Ax}=0\}\subset \mathbb{R}^n
  $$ 
  $\text{Dim}(\mathcal{N}(\bm{A}))=n-r$.

  \item[$\cdot$] $\mathcal{N}(\pmb{A}^{\top}):=\{\bm{x}: \bm{\pmb{A}^{\top}x}=0\}\subset \mathbb{R}^m$. $\text{Dim}(\mathcal{N}(\pmb{A}^{\top}))=m-r$.
\end{itemize}
\section{}
\begin{itemize}
  \item[$\cdot$] Eigval $\lambda$, eigvec $\bm{v}$ such that: $\bm{A} \bm{v} = \lambda \bm{v}$.
  \item[$\cdot$]$c \bm{v}$ is also an eigvec corresponding to $\lambda$, but only linearly indep. eigvecs are counted.
\end{itemize}

\begin{itemize}
  \item[$\cdot$] $\bm{A}$ is $n\times n$ square, then the characteristic polynomial of $\bm{A}$ is defined as.
  $$
  P_{\bm{A}}(t):=\det(t \bm{I} - \bm{A})
  $$
  \item[$\cdot$]$\lambda$ is eigval of $\bm{A}$ $\iff P_{\bm{A}}(\lambda) = 0\iff \lambda$ is root of $P_{\bm{A}}(t)=0$.\\
  \textit{Proof.~~} ($\Rightarrow$) $\lambda$ is eigval of $\bm{A}$: $(\lambda \bm{I}-\bm{A})\bm{v}=0$. \\
  $\bm{v}$ is an eigvec of $\bm{A}$, so $\bm{v}\ne 0$ $\Rightarrow \bm{v}\in \mathcal{N}(\bm{A})$.
  Hence $\text{Dim}(\mathcal{N}(\bm{A}))>0$ $\Rightarrow r<n$. $\square$

  \item[$\cdot$] $P_{\bm{A}}(t)$ can be represented as, sence we know the roots, 
  $$
  P_{\bm{A}}(t) = \prod_{j=1}^n (t- \lambda_j)
  $$
  \item[$\cdot$] $\bm{D}$ being diagonal matrix, then $P_{\bm{D}} = \prod_{j=1}^n (t- d_j)$, which implies that $\lambda_j = d_j$, diagonal entries. Moreover, $\bm{D}\bm{e}_j = d_j \bm{e}_j$, so the $j$-th eigvec is $\bm{e}_j$ unit vec.
  \item[$\cdot$] $\bm{L}$ being lower triangular, then $P_{\bm{L}} = \prod_{j=1}^n (t- L_{jj})$, which implies that $\lambda_j = d_j$. The last col of lower tri is $L_{nn} \bm{e}_n$, therefore $\bm{L}\bm{e}_n = L_{nn}\bm{e}_n$, i.e. the last eigvec is $\bm{e}_n$ unit vector. We can not tell about other eigvecs. Similar for $\bm{U}$. $\bm{U}\bm{e}_1 = U_{11}\bm{e}_1$, i.e. the first eigvec of $\bm{U}$ is $\bm{e}_1$ unit vector.
\end{itemize}

\begin{itemize}
  \item[$\cdot$] $\lambda(\bm{A})$ is the set of all eigvals of $\bm{A}$, it is acturally the spectrum of $\bm{A}$.
  \item[$\cdot$] If $\lambda \in \lambda(\bm{A})$ is a root of multiplicity $m_{\lambda}$ of $P_{\bm{A}}(t)=0$, define $m_{\lambda}$ as multiplicity of eigval $\lambda$.
  \item[$\cdot$] Square matrix $\bm{A}$ has exactly $n$ eigvals ($\lambda\in \mathbb{C}$), counted with multiplicity, i.e.
  $$
  \sum_{\lambda \in \lambda(\bm{A})} m_{\lambda} = n
  $$
  This is because, due to fundamental principal of algebra, $P_{\bm{A}}(t)=0$ has exactly $n$ roots, counted with multiplicity.
\end{itemize}

\begin{itemize}
  \item[$\cdot$] Eigenspace (eigspace) of $\lambda$: $V_{\lambda}:=\{\bm{v: \bm{A}\bm{v}=\lambda \bm{v}}\}$, i.e. the set of all eigvecs corresponding to $\lambda$. $\text{Dim}(V_{\lambda})$ is the number of linearly indep. eigvecs corresponding to $\lambda$. We have
  $$
  1\leq \text{Dim}(V_{\lambda})\leq m_{\lambda}
  $$
  \item[\textit{Thm.~}] eigvecs corresponding to \textit{different} eigvals of $\bm{A}$ are linearly indep.\\
  \textit{Proof.~~} Let $\bm{v}_1, ..., \bm{v}_p$ correspond to different eigvals of $\bm{A}$: $\lambda_1, ..., \lambda_p$ ($p\leq n$).\\
  It suffices to show
  $$
  c_1 \bm{v}_1 + ... + c_p \bm{v}_p= \bm{0} \Rightarrow c_1=...=c_p=0
  $$
  Show by contradiction: suppose otherwise, i.e. 
  $$
  (\dag): c_1 \bm{v}_1 + ... + c_{p-1} \bm{v}_{p-1}= \bm{v}_p
  $$
  Apply $\bm{A}$ both sides:
  \begin{equation}
  \begin{split}
    \bm{A}(c_1 \bm{v}_1 + ... + c_{p-1} \bm{v}_{p-1}) &= \bm{A} \bm{v}_p\\
    c_1 \lambda_1 \bm{v}_1 + ... + c_{p-1} \lambda_{p-1} \bm{v}_{p-1} &= \lambda_p \bm{v}_p
  \end{split}
  \end{equation}
  $(\dag)\times \lambda_p$, substracted from last equation:
  $$
  \sum_{j=1}^{p-1}c_j (\lambda_1 - \lambda_p)\bm{v}_j = \lambda_p \bm{v}_p - \lambda_p \bm{v}_p = 0
  $$
  Which is not possible since $\exists c_j \ne 0$, and $(\lambda_j - \lambda_p)\ne 0~\forall j$. Contradiction. $\square$
\end{itemize}

\begin{itemize}
  \item[$\cdot$] $\bm{A}$ singular $\iff$ $0\in \lambda(\bm{A})$.\\
  \textit{Proof.~~} $\det(0 \bm{I}-\bm{A})=0 \iff$ $\bm{A}$ singular. $\square$

  \item[$\cdot$] $\bm{A}$ invertible. $\lambda \in \lambda(\bm{A})$, $\bm{v}\in V_{\lambda}(\bm{A})$. Then $\frac{1}{\lambda}\in \lambda(\bm{A}^{-1})$, $\bm{v}\in V_{\frac{1}{\lambda}}(\bm{A}^{-1})$.\\
  \textit{Proof.~~} $\bm{Av}=\lambda \bm{v}$ $\Rightarrow$ $\bm{A}^{-1}\bm{Av} = \lambda \bm{A}^{-1} \bm{v}$ $\Rightarrow$ $\frac{1}{\lambda} \bm{v} = \bm{A}^{-1} \bm{v}$. 

  \item[$\cdot$] $\bm{A}$ has eigtuple $(\lambda, \bm{v})$, then $\bm{A}^k$ with $(\lambda^k, v)$. $P(\bm{A})$ is a polynomial of $\bm{A}$, has eigtuple $(P(\lambda), \bm{v})$.

\end{itemize}
%--------------------------------------------------------------------
\fbox{
	\parbox{\boxwidth}{
	1
	}
}
%--------------------------------------------------------------------


\end{document}