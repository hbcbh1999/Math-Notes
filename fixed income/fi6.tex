\documentclass[a4paper, 10pt]{article}    
\usepackage{geometry}       
\geometry{a4paper}
\geometry{margin=1in} 
\usepackage{paralist}
  \let\itemize\compactitem
  \let\enditemize\endcompactitem
  \let\enumerate\compactenum
  \let\endenumerate\endcompactenum
  \let\description\compactdesc
  \let\enddescription\endcompactdesc
  \pltopsep=\medskipamount
  \plitemsep=1pt
  \plparsep=1pt
\usepackage[english]{babel}
\usepackage[utf8]{inputenc}
\usepackage{color}
\usepackage{bbm, bm}
\usepackage{amsmath, amssymb, amsthm, mathrsfs}
\usepackage{booktabs, tikz, array}
% Small bullet
\usepackage{lscape}
\usepackage{graphicx}
\usepackage{rotating}

\usepackage{float}
\usepackage{caption} 
\renewcommand{\arraystretch}{1.4}
\newcolumntype{L}{>{\arraybackslash}m{12cm}}

\newcommand\indep{\protect\mathpalette{\protect\indeP}{\perp}}
\def\indeP#1#2{\mathrel{\rlap{$#1#2$}\mkern2mu{#1#2}}}

\pagestyle{headings}
\newcommand{\boxwidth}{430pt}

\theoremstyle{definition}
\newtheorem{problem}{Problem}

\newtheoremstyle{hSol}
  {1.0pt}% Space above
  {1.0pt}% Space below
  {}% bodyfont
  {}% indent
  {\bfseries}% thm head font
  {.}% punctuation after thm head
  { }% Space after thm head
  {}% thm head spec

\theoremstyle{hSol}
\newtheorem*{solution}{Solution}



\title{\textbf{Fixed Income Assignment V}}
\author{Ze Yang~~~~(zey@andrew.cmu.edu)}

\begin{document}
\maketitle


\begin{table}[h]
\vspace{-10pt}
\caption{\textit{Nomenclatures}}
\vspace{3pt}
\centering
\def\arraystretch{1.15}
\begin{tabular}{lL}
\hline
Notation & \hspace{4.6cm} Description \\ 
\hline
$\bm{w}$ & The $N$-periods coin toss path, $\bm{w} = (w_1~~w_2~~...~~w_N)^{\top}$, where $w_n$ is the outcome for single period $n$.\\
$\psi_n$ & The partial path till $i$-th period. I.e. the event of all $\bm{w}$ that has same trajectory until $i$: $\psi_n=\{\bm{w}: w_k=\hat{w}_k, k =1,2,...,n\}$\\
$R_n$ & The short rate process, $R_n = R_n(\psi_n)$. \\
$P(B)$ & The price of bond $B$, $P(B)=\sum_{(t,C_t) \in B}C_td(t)=\langle d(\bm{t_B}), \bm{C}_B \rangle = \bm{C}_B^{\top}\bm{d}(\bm{t}_B)$.  \\
$\tilde{p}(w), \tilde{\mathbb{P}}\left(w\right)$ & Risk neutral probabilities. In the coin toss space, we refer to $\tilde{p}(H)$ as $\tilde{p}$, $\tilde{p}(T)$ as $\tilde{q}$.\\
$\#H(\psi_n)$, $\#T(\psi_n)$ & The number of heads (tails) in partial path $\psi_n$. \\
$d(i, j)$ & The discount factor from period $i$ to $j$. \\
$V_n, C_n$ & Price of security at $n$ after cash payment; the cash payment at time $n$.\\
$\tilde{\mathbb{E}}_i\left[V_{j}\right](\psi_i)$ & The expectation of security payoff of period $j$ conditional on period $i<j$; it's a random variable with respect to partial path $\psi_i$. Tilde means the risk neutral probability measure.\\
\hline 
\end{tabular}
\label{tab:Nomen}
\end{table}




\noindent\rule{16cm}{0.4pt}
%///////////////////////////////////////////////////////////////////////

\begin{problem} 
\end{problem}
\begin{solution} \textbf{(a)} For security 1, we have to convert the price of underlying from $m=3$ to time 2, in order to calculate $For_{i, 2}$ and $Fut_{i,2}$.
$$
P_2 = \frac{P_3}{1+R_2}
$$

\begin{table}[h]
\vspace{-1pt}
\caption{\textit{Futures \& Forwards I}}
\vspace{-7pt}
\centering
\def\arraystretch{1.15}
\begin{tabular}{|r|ccccc|}
\hline
$n$ & $R_n$ & $D_{n}$ & $P_n$ & $D_{n}P_n$ & $\tilde{\mathbb{E}}_{n}\left[P_2\right]$\\ 
\hline
2 & 0.08 & 0.8850 & 740.741 & 655.546&\\
2 & 0.07 & 0.8850 & 654.206 & 578.964&\\
2 & 0.07 & 0.8968 & 654.206 & 586.668&\\
2 & 0.05 & 0.8968 & 476.190 & 427.031&\\
\hline
1 & 0.066 & 0.9434 & && 697.474 \\
1 & 0.052 & 0.9434 & &&565.198 \\
\hline
0 & 0.06 & 1 & &&631.336 \\
\hline
\end{tabular}
\label{tab:ff1}
\end{table}
~\\
Hence, as shown in the table we have:
\begin{equation}
  \begin{split}
    &Fut_{0,2} = \tilde{\mathbb{E}}\left[P_2\right] = 631.336\\
    &Fut_{1,2}(H) = \tilde{\mathbb{E}}_1\left[P_2\right](H) = 697.474,~~~~Fut_{1,2}(T) = \tilde{\mathbb{E}}_1\left[P_2\right](T) = 565.198
  \end{split}
\end{equation}
And $\tilde{\mathbb{E}}\left[P_2D_2\right] = (655.546+578.964+586.668+427.031)/4 = 562.052136$. $\tilde{\mathbb{E}}\left[D_2\right] = (0.8850+0.8968)/2 = 0.890876$. Therefore
$$
For_{0,2} = \frac{\tilde{\mathbb{E}}\left[P_2D_2\right]}{\tilde{\mathbb{E}}\left[D_2\right]} = 630.898438
$$
\textbf{(b)}
\begin{table}[h]
\vspace{-1pt}
\caption{\textit{Futures \& Forwards II}}
\vspace{-7pt}
\centering
\def\arraystretch{1.15}
\begin{tabular}{|r|ccccc|}
\hline
$n$ & $R_n$ & $D_{n}$ & $P_n$ & $D_{n}P_n$ & $\tilde{\mathbb{E}}_{n}\left[P_2\right]$\\ 
\hline
2 & 0.08 & 0.8850 & 800 & 707.990&\\
2 & 0.07 & 0.8850 & 700 & 619.491&\\
2 & 0.07 & 0.8968 & 700 & 627.735&\\
2 & 0.05 & 0.8968 & 500 & 448.382&\\
\hline
1 & 0.066 & 0.9434 & && 750 \\
1 & 0.052 & 0.9434 & && 600 \\
\hline
0 & 0.06 & 1 & && 675 \\
\hline
\end{tabular}
\label{tab:ff2}
\end{table}
\begin{equation}
  \begin{split}
    &Fut_{0,2} = \tilde{\mathbb{E}}\left[P_2\right] = 675\\
    &Fut_{1,2}(H) = \tilde{\mathbb{E}}_1\left[P_2\right](H) = 750,~~~~Fut_{1,2}(T) = \tilde{\mathbb{E}}_1\left[P_2\right](T) = 600
  \end{split}
\end{equation}
$$
For_{0,2} = \frac{\tilde{\mathbb{E}}\left[P_2D_2\right]}{\tilde{\mathbb{E}}\left[D_2\right]} = \frac{(707.990+619.491+627.735+448.382)/4}{0.890876} = 674.50425
$$
\end{solution}

\noindent\rule{16cm}{0.4pt}
%///////////////////////////////////////////////////////////////////////

\begin{problem} 
\end{problem}
\begin{solution} 

\begin{table}[h]
\vspace{-1pt}
\caption{\textit{Futures \& Forwards III}}
\vspace{-7pt}
\centering
\def\arraystretch{1.15}
\begin{tabular}{|r|ccccccc|}
\hline
$n$ & $R_n$ & $D_{n}$ & $P_n$ & $D_{n}P_n$ & $\tilde{\mathbb{E}}_{n}\left[P_2\right]$ & $\tilde{\mathbb{E}}_{n}\left[D_2P_2\right]$ & $\tilde{\mathbb{E}}_{n}\left[D_2P_2\right]/ \tilde{\mathbb{E}}_n\left[D_2\right]$\\ 
\hline
2 & 0.06 & 0.9027 & 94.3396 & 87.6317&&&\\
2 & 0.05 & 0.9027 & 95.2381 & 86.7970&&&\\
2 & 0.05 & 0.9114 & 95.2381 & 85.9744&&&\\
2 & 0.04 & 0.9114 & 96.1538 & 85.1633&&&\\
\hline
1 & 0.055 & 0.9524 & && 95.6960 & 87.2144 & 96.6117\\
1 & 0.045 & 0.9524 & && 94.7889 & 85.5688 & 93.8904\\
\hline
0 & 0.05 & 1 & && 95.2424 & 86.3916 & 95.2446\\
\hline
\end{tabular}
\label{tab:ff3}
\end{table}

As shown in the table:
\begin{equation}
  \begin{split}
    &Fut_{0,2} = \tilde{\mathbb{E}}\left[P_2\right] = 95.24241482\\
    &Fut_{1,2}(H) = \tilde{\mathbb{E}}_1\left[P_2\right](H) = 95.6960,~~~~Fut_{1,2}(T) = \tilde{\mathbb{E}}_1\left[P_2\right](T) = 94.7889
  \end{split}
\end{equation}
\begin{equation}
  \begin{split}
    &For_{0,2} = \frac{\tilde{\mathbb{E}}\left[D_2P_2\right]}{\tilde{\mathbb{E}}\left[D_2\right]} = \frac{86.39160}{0.90705} = 95.2445746\\
    &For_{1,2}(H) = \frac{\tilde{\mathbb{E}}_1\left[D_2P_2\right](H)}{\tilde{\mathbb{E}}_1\left[D_2\right](H)} = \frac{87.2144}{0.9027} = 96.6117216\\
    &For_{1,2}(T) = \frac{\tilde{\mathbb{E}}_1\left[D_2P_2\right](T)}{\tilde{\mathbb{E}}_1\left[D_2\right](T)} = \frac{85.5688}{0.9114} = 93.89038634
  \end{split}
\end{equation}

\end{solution}

\noindent\rule{16cm}{0.4pt}
%///////////////////////////////////////////////////////////////////////

\begin{problem} 
\end{problem}
\begin{solution} ~\\
% Table generated by Excel2LaTeX from sheet 'Sheet2'
\begin{table}[H]
  \centering
  \caption{\textit{Callable Bond I}}
  \vspace{-7pt}
  \def\arraystretch{1.15}
    \begin{tabular}{|r|rrrr|}
    \hline
    $n$ & $R_n$ & $C_n$ & $V_n$ & $C_n+V_n$\\ 
    \hline
    3     &       & 1061.004 &       & 1061.00407 \\
    3     &       & 1061.004 &       & 1061.00407 \\
    3     &       & 1061.004 &       & 1061.00407 \\
    \hline
    2     & 0.07  & 61.00407 & 991.592586 & 1052.59665 \\
    2     & 0.06  & 61.00407 & 1000  & 1061.00407 \\
    2     & 0.05  & 61.00407 & 1000  & 1061.00407 \\
    \hline
    1     & 0.065 & 61.00407 & 992.300807 & 1053.30487 \\
    1     & 0.055 & 61.00407 & 1005.69106 & 1066.69513 \\
    \hline
    0     & 0.06  & 0     & 1000  & 1000 \\
    \hline
    \end{tabular}%
  \label{tab:cb1}%
\end{table}%
~\\
Computed with solver: $q^{(4)} = 0.06100407$.

% Table generated by Excel2LaTeX from sheet 'Sheet2'
\begin{table}[htbp]
  \centering
    \caption{\textit{Callable Bond II}}
  \vspace{-7pt}
  \def\arraystretch{1.15}
    \begin{tabular}{|r|rrrr|}
    \hline
    $n$ & $R_n$ & $C_n$ & $V_n$ & $C_n+V_n$\\
     \hline
    3     &       & 1062.6889 &       & 1062.6889 \\
    3     &       & 1062.6889 &       & 1062.6889 \\
    3     &       & 1062.6889 &       & 1062.6889 \\
    2     & 0.07  & 62.68894 & 993.16723 & 1055.8562 \\
    2     & 0.06  & 62.68894 & 1000  & 1062.6889 \\
    2     & 0.05  & 62.68894 & 1000  & 1062.6889 \\
    1     & 0.065 & 62.68894 & 994.62212 & 1057.3111 \\
    1     & 0.055 & 62.68894 & 1000  & 1062.6889 \\
    \hline
    0     & 0.06  & 0     & 1000  & 1000 \\
    \hline
    \end{tabular}%
  \label{tab:cb2}%
\end{table}%
~\\
Computed with solver: $q^{(5)} = 0.06268894$.


\end{solution}

\noindent\rule{16cm}{0.4pt}
%///////////////////////////////////////////////////////////////////////

\begin{problem} 
\end{problem}
\begin{solution} \textbf{(a)} At time 0, the mortgage has "fair price", i.e. the price is equal to outstanding principle. Adopting same notations as the lecture notes, we have
\begin{equation}
  P_0 = A \sum_{i=1}^3 \lambda^i = A \sum_{i=1}^3 \frac{1}{(1+y)^i}
\end{equation}
We can solve $A$ with $P_0=100$, $y=0.1$: $A=40.21148$. The cashflow profile of IO and PO strips is calculated in the table below.
\begin{table}[H]
  \centering
  \caption{Mortgage Cashflow Profile}
    \vspace{-7pt}
  \def\arraystretch{1.15}
    \begin{tabular}{|r|rrrr|}
    \hline
    $n$ & $A$ & $I_n$ & $P_n$ & $P_{remain}$\\
    \hline
    0     &       &       &       & 100 \\
    1     & 40.2114804 & 10    & 30.21148036 & 69.78851964 \\
    2     & 40.2114804 & 6.97885196 & 33.2326284 & 36.55589124 \\
    3     & 40.2114804 & 3.65558912 & 36.55589124 & 0 \\
    \hline
    \end{tabular}%
  \label{tab:mtge}%
\end{table}%
~\\
The price of IO and PO is calculated in two tables below.

% Table generated by Excel2LaTeX from sheet 'Sheet4'
\begin{table}[H]
  \centering
  \caption{\textit{IO Strip I}}
  \vspace{-7pt}
  \def\arraystretch{1.15}
  \begin{tabular}{|r|rrrr|}
  \hline
  $n$ & $R_n$ & $C_n$ & $V_n$ & $C_n+V_n$\\
    \hline
    3     &       & 3.65558912 &       & 3.65558912 \\
    3     &       & 3.65558912 &       & 3.65558912 \\
    3     &       & 3.65558912 &       & 3.65558912 \\
    \hline
    2     & 0.14  & 6.97885196 & 3.20665713 & 10.1855091 \\
    2     & 0.1   & 6.97885196 & 3.32326284 & 10.3021148 \\
    2     & 0.06  & 6.97885196 & 3.44866898 & 10.4275209 \\
    \hline
    1     & 0.12  & 10    & 9.14626067 & 19.1462607 \\
    1     & 0.08  & 10    & 9.59705359 & 19.5970536 \\
    \hline
    0     & 0.1   &       & 17.6105974 & 17.6105974 \\
    \hline
    \end{tabular}%
  \label{tab:addlabel}%
\end{table}%

% Table generated by Excel2LaTeX from sheet 'Sheet4'
\begin{table}[H]
  \centering
  \caption{\textit{PO Strip I}}
  \vspace{-7pt}
  \def\arraystretch{1.15}
  \begin{tabular}{|r|rrrr|}
  \hline
  $n$ & $R_n$ & $C_n$ & $W_n$ & $C_n+W_n$\\
    \hline
    3     &       & 36.55589124 &       & 36.55589124 \\
    3     &       & 36.55589124 &       & 36.55589124 \\
    3     &       & 36.55589124 &       & 36.55589124 \\
    \hline
    2     & 0.14  & 33.2326284 & 32.0665713 & 65.29919966 \\
    2     & 0.1   & 33.2326284 & 33.2326284 & 66.4652568 \\
    2     & 0.06  & 33.2326284 & 34.4866898 & 67.71931825 \\
    \hline
    1     & 0.12  & 30.21148036 & 58.8234181 & 89.03489842 \\
    1     & 0.08  & 30.21148036 & 62.1224884 & 92.33396881 \\
    \hline
    0     & 0.1   &       & 82.4403942 &  82.4403942\\
    \hline
    \end{tabular}%
  \label{tab:addlabel}%
\end{table}%
~\\
We have: $V_0=  17.6105974$; $W_0 = 82.4403942$.\\
~\\
\textbf{(b)} The price of IO and PO is calculated in two tables below.
% Table generated by Excel2LaTeX from sheet 'Sheet4'
\begin{table}[H]
  \centering
  \caption{\textit{IO Strip II}}
  \vspace{-7pt}
  \def\arraystretch{1.15}
  \begin{tabular}{|r|rrrr|}
  \hline
  $n$ & $R_n$ & $C_n$ & $V_n$ & $C_n+V_n$\\
    \hline
    3     &       & 2.96  &       & 2.96 \\
    3     &       & 2.96  &       & 2.96 \\
    3     &       & 2.63  &       & 2.63 \\
    3     &       & 1.76  &       & 1.76 \\
    \hline
    2     & 0.14  & 6.28  & 2.59649123 & 8.87649123 \\
    2     & 0.1   & 6.28  & 2.69090909 & 8.97090909 \\
    2     & 0.1   & 5.58  & 2.39090909 & 7.97090909 \\
    2     & 0.06  & 5.58  & 1.66037736 & 7.24037736 \\
    \hline
    1     & 0.12  & 10    & 7.96758943 & 17.9675894 \\
    1     & 0.08  & 10    & 7.04226225 & 17.0422622 \\
    \hline
    0     & 0.1   &       & 15.9135689 & 15.9135689 \\
    \hline
    \end{tabular}%
  \label{tab:addlabel}%
\end{table}%

% Table generated by Excel2LaTeX from sheet 'Sheet4'
\begin{table}[H]
  \centering
  \caption{\textit{PO Strip II}}
  \vspace{-7pt}
  \def\arraystretch{1.15}
  \begin{tabular}{|r|rrrr|}
  \hline
  $n$ & $R_n$ & $C_n$ & $W_n$ & $C_n+W_n$\\
    \hline
    3     &       & 29.61 &       & 29.61 \\
    3     &       & 29.61 &       & 29.61 \\
    3     &       & 26.32 &       & 26.32 \\
    3     &       & 17.55 &       & 17.55 \\
    \hline
    2     & 0.14  & 33.2  & 25.9736842 & 59.17368421 \\
    2     & 0.1   & 33.2  & 26.9181818 & 60.11818182 \\
    2     & 0.1   & 29.51 & 23.9272727 & 53.43727273 \\
    2     & 0.06  & 38.28 & 16.5566038 & 54.83660377 \\
    \hline
    1     & 0.12  & 37.19 & 53.2552973 & 90.44529733 \\
    1     & 0.08  & 44.17 & 50.1267947 & 94.29679468 \\
    \hline
    0     & 0.1   &       & 83.9736782 & 83.97367819 \\
    \hline
    \end{tabular}%
  \label{tab:addlabel}%
\end{table}%
~\\
We have: $V_0=  15.9135689 $; $W_0 = 83.97367819$.\\
~\\



\end{solution}

\noindent\rule{16cm}{0.4pt}
%///////////////////////////////////////////////////////////////////////

\begin{problem} 
\end{problem}
\begin{solution} Suppose the (single) callable times are $\tau \in \mathcal{T}$. At $\tau$, the buyer should put back the bond if the post-coupon price of the bond is smaller than par. We have
\begin{equation}
  \begin{split}
  V_n 
  &=\begin{cases}
  0 & n=N\\
  \frac{1}{1+R_{n}}\left(C_{n+1} + \tilde{p}V_{n+1}(\psi_n, H) + \tilde{q}V_{n+1}(\psi_n, T)\right) & 0\leq n < N, n\notin \mathcal{T}\\
  \max\{F, \frac{1}{1+R_{n}}\left(C_{n+1} + \tilde{p}V_{n+1}(\psi_n, H) + \tilde{q}V_{n+1}(\psi_n, T)\right)\} & n \in \mathcal{T} \\
  \end{cases}
  \end{split}
\end{equation}
The calculation is presented in the tables below.

% Table generated by Excel2LaTeX from sheet 'Sheet3'
\begin{table}[H]
  \centering
  \caption{\textit{Putable Bond I}}
  \vspace{-7pt}
  \def\arraystretch{1.15}
  \begin{tabular}{|r|rrrr|}
  \hline
  $n$ & $R_n$ & $C_n$ & $V_n$ & $C_n+V_n$\\
    \hline
    3     &       & 1060  &       & 1060 \\
    3     &       & 1060  &       & 1060 \\
    3     &       & 1060  &       & 1060 \\
    \hline
    2     & 0.07  & 60    & 1000  & 1060 \\
    2     & 0.06  & 60    & 1000  & 1060 \\
    2     & 0.05  & 60    & 1009.52381 & 1069.52381 \\
    \hline
    1     & 0.065 & 60    & 995.3051643 & 1055.305164 \\
    1     & 0.055 & 60    & 1009.25299 & 1069.25299 \\
    \hline
    0     & 0.06  & 0     & 1002.150073 & 1002.150073 \\
    \hline
    \end{tabular}%
  \label{tab:put1}%
\end{table}%
~\\ 
\textbf{(a)} We have $V_0^{[\#1]} = 1002.150073$.\\

% Table generated by Excel2LaTeX from sheet 'Sheet3'
\begin{table}[H]
  \centering
  \caption{\textit{Putable Bond II}}
  \vspace{-7pt}
  \def\arraystretch{1.15}
  \begin{tabular}{|r|rrrr|}
  \hline
  $n$ & $R_n$ & $C_n$ & $V_n$ & $C_n+V_n$\\
    \hline
    3     &       & 1060  &       & 1060 \\
    3     &       & 1060  &       & 1060 \\
    3     &       & 1060  &       & 1060 \\
    \hline
    2     & 0.07  & 60    & 1000  & 1060 \\
    2     & 0.06  & 60    & 1000  & 1060 \\
    2     & 0.05  & 60    & 1009.52381 & 1069.52381 \\
    \hline
    1     & 0.065 & 60    & 1000  & 1060 \\
    1     & 0.055 & 60    & 1009.25299 & 1069.25299 \\
    \hline
    0     & 0.06  & 0     & 1004.364618 & 1004.364618 \\
    \hline
    \end{tabular}%
  \label{tab:pb2}%
\end{table}%
~\\ 
\textbf{(b)} We have $V_0^{[\#2]} = 1004.364618$.\\


\end{solution}

\noindent\rule{16cm}{0.4pt}
%///////////////////////////////////////////////////////////////////////

\begin{problem} 
\end{problem}
\begin{solution} 
\textbf{(a)} The DV01 of the ZCB is
\begin{equation}
  DV01_{ZCB} = \frac{1}{10^4}\frac{FT}{(1+\frac{y}{2})^{2T+1}} = \frac{1}{10^4}\frac{5\times10^5\times 30}{(1+\frac{0.05}{2})^{61}} = 332.6101286
\end{equation}
And use 30-yr par coupon bond to match this DV01:
$$
DV01_{ZCB} = DV01_{pc} = \frac{F_{pc}}{10^4 y} \left(1-\frac{1}{(1+\frac{y}{2})^{2T}}\right)
$$
$$
F_{pc} = \frac{10^4 y\times DV01_{ZCB}}{(1-\frac{1}{(1+\frac{y}{2})^{2T}})} = 332.6101286\times  \frac{10^4 \times 0.05}{(1-\frac{1}{(1+\frac{0.05}{2})^{60}})} = 215234.27
$$
So we long $215234.27$ face of the par bond.\\
~\\
\textbf{(b)}
\begin{figure}[H]
  \centering
  \captionsetup{justification=centering}
  \caption{\label{fig:span}Spread Sheet Calculation}
  \vspace{-10pt}
  \includegraphics[scale=0.6]{figures/fig2.png}
\end{figure}
We use the spreadsheet posted on Canvas to calculate the DV01 for each par coupon bond. And use formula:
$$
DV01(T) = \frac{F}{10^4 y_{pc}(T)} \left(1-\frac{1}{(1+\frac{y_{pc}(T)}{2})^{2T}}\right)
$$
to solve face $F$ for each par coupon bond that matches corresponding $DV01$. Let $F_1, F_2, F_3, F_4$ be the face of 2-, 5-, 10-, and 30-year par coupon bond, we have
$$
F_1 = -20065.02,~~~F_2=-33162.67,~~~F_3= -209062.01,~~~F_4=332282.79
$$
So we Short 20065.02 of the 2-yr bond, short 33162.67 of the 5-yr bond, short 209062.01 of the 10-yr bond, and Long 332282.79 of the 30-yr bond.
\end{solution}





\end{document}